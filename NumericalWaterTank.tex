%% PRE EDITION
\documentclass[a4paper]{article}
\usepackage[utf8]{inputenc}
\usepackage[T1]{fontenc}
\usepackage[french]{babel}
\usepackage{soul}
\usepackage[pdftex]{graphicx}

\usepackage{amsfonts}
\usepackage{amsthm}
\usepackage{amsmath}
\usepackage{amssymb}
\usepackage{mathrsfs}
\usepackage{booktabs}
\usepackage{siunitx}
\usepackage{thmtools}
\usepackage{ulem}

%% LAYOUT TITLE
\usepackage[explicit]{titlesec}
\usepackage{xcolor}
\usepackage{times}
\usepackage{tikz}
\usepackage{lipsum}
\titleformat{\subsection}
  {\color{blue!70}\large\sffamily\bfseries}
  {}
  {0em}
  {\colorbox{blue!10}{\parbox{\dimexpr\linewidth-2\fboxsep\relax}{\arabic{section}.\arabic{subsection}. #1}}}
  []

%% LAYOUT MATHS
\declaretheoremstyle[
    bodyfont=\normalfont\color{red},
    headfont=\color{red}
]{styleattention}

\declaretheoremstyle[
    spacebelow=1em
]{styleremarque}

\declaretheoremstyle[
    spaceabove=-6pt, 
    spacebelow=6pt, 
    headfont=\normalfont\bfseries, 
    bodyfont = \normalfont,
    postheadspace=1em, 
    qed=$\Box$, 
    headpunct={$\rhd$}
]{mystyle} 

\declaretheorem[thmbox=M,numberwithin=section,title=Définition]{definition}
\declaretheorem[thmbox=M,sibling=definition]{proposition}
\declaretheorem[thmbox=M,sibling=definition]{corollaire}
\declaretheorem[thmbox=M,sibling=definition,title=Théorème]{theoreme}
\declaretheorem[thmbox=M,sibling=definition]{lemme}
\declaretheorem[thmbox=M,sibling=definition,title=Propriété]{propriete}
\declaretheorem[thmbox=M,sibling=definition,title=Propriétés]{proprietes}
\declaretheorem[style=styleremarque,sibling=definition,title=Remarque]{remarque}
\declaretheorem[style=styleattention,title=À revoir]{Arevoir}
\declaretheorem[name={}, style=mystyle, unnumbered]{preuve}


\renewcommand\qedsymbol{$\blacksquare$}

%% NEW COMMAND
% solution
\newcommand{\y}{y}
% concentration monomers
\newcommand{\cmono}{c}
\newcommand{\pol}{a}
\newcommand{\dep}{b}
\newcommand{\mass}{\mathrm{M}}

\usepackage{geometry}
\geometry{hmargin=3cm,vmargin=2.5cm}

\usepackage{tabularx}
\usepackage{float}

\title{Numerical solution of water tank stabilization}
\author{CDV, CZ, AH}

%% DEBUT DE REDACTION
\begin{document}

\maketitle

%%%%%%%%%%%%%%%%%%%%%%%%%%%%%%%%%%%%%%%%%%%%%%%%%%%%%%%%%%%%%%%%%%%%%%%%%%%%%%%%%%%%%%%%%%%%%%%%%%%%%%%%%%%%%%%%%%%%%%%%%%%%%%%%%
%%%%%%%%%%%%%%%%%%%%%%%%%%%%%%%%%%%%%%%%%%%%%%%%%%%%%%%%%%%%%%%%%%%%%%%%%%%%%%%%%%%%%%%%%%%%%%%%%%%%%%%%%%%%%%%%%%%%%%%%%%%%%%%%%
%%%%%%%%%%%%%%%%%%%%%%%%%%%%%%%%%%%%%%%%%%%%%%%%%%%%%%%%%%%%%%%%%%%%%%%%%%%%%%%%%%%%%%%%%%%%%%%%%%%%%%%%%%%%%%%%%%%%%%%%%%%%%%%%%
%%%%%%%%%%%%%%%%%%%%%%%%%%%%%%%%%%%%%%%%%%%%%%%%%%%%%%%%%%%%%%%%%%%%%%%%%%%%%%%%%%%%%%%%%%%%%%%%%%%%%%%%%%%%%%%%%%%%%%%%%%%%%%%%%
%%%%%%%%%%%%%%%%%%%%%%%%%%%%%%%%%%%%%%%%%%%%%%%%%%%%%%%%%%%%%%%%%%%%%%%%%%%%%%%%%%%%%%%%%%%%%%%%%%%%%%%%%%%%%%%%%%%%%%%%%%%%%%%%%

%%%%%%%%%%%%%%%%%%%%%%%%%%%%%%%%%%%%%%%%%%%%%%%%%%%%%%%%%%%%%%%%%%%%%%%%%%%%%%%%%%%%%%%%%%%%%%%%%%%%%%%%%%%%%%%%%%%%%%%%%%%%%%%%%
\section{Mathematical settings}
%%%%%%%%%%%%%%%%%%%%%%%%%%%%%%%%%%%%%%%%%%%%%%%%%%%%%%%%%%%%%%%%%%%%%%%%%%%%%%%%%%%%%%%%%%%%%%%%%%%%%%%%%%%%%%%%%%%%%%%%%%%%%%%%%

The objective of this document is to explain the numerical choices made 
to simulate the stabilization of a water tank by a back stepping method.

We therefore consider the couple $(h,v)$ two functions of $L^2(0,L)$
solutions of the 1D Saint-Venant equations:

\[
\partial_t 
\begin{bmatrix}
	h \\
	v
\end{bmatrix} 
+  
\begin{bmatrix}
	0 & H^\gamma \\
	1 7 0
\end{bmatrix} 
\partial_x
\begin{bmatrix}
	h \\
	v
\end{bmatrix} 
+
\begin{bmatrix}
	0 & -\gamma \\
	0 & 0
\end{bmatrix} 
\begin{bmatrix}
	h \\
	v
\end{bmatrix} 
=
-u(t)
 \begin{bmatrix}
	0 \\
	1
\end{bmatrix} 
\]

With boundary conditions and conservation of mass, I do not copy everything at this stage.

In order to compute the control $u$ that stabilizes the water tank,
we define the following transformation

\[
\begin{bmatrix}
	\xi_1 \\
	\xi_2
\end{bmatrix} 
=
\begin{bmatrix}
	\sqrt{\frac{1}{H^\gamma}} & 1 \\
	- \sqrt{\frac{1}{H^\gamma}} & 1
\end{bmatrix} 
\begin{bmatrix}
	h \\
	v
\end{bmatrix} 
\]


In this document we will solve the differential equation satisfied by the variable $\xi = (\xi_1,\xi_2)$


\begin{equation}
	\label{def:transport}
	\begin{cases}
		\partial_t \xi + \Lambda \partial_x \xi + \delta(x) J \xi = \text{TO BE DETERMINED} \\
		\xi_1 (0,t) = -\xi_2 (0,t) ,\\
		\xi_2 (L,t) = -\xi_2(L,t).
	\end{cases}
\end{equation}

We notice that these boundaries'conditions allows us to consider the periodic system $(\xi_1 (x,t), -\xi_2(-x,t))$.

Our strategy to solve this problem will be to consider the derivation operators as operators of finite differences. 
Then we will calculate the eigenvalues associated with the operator of the dynamics, a
nd it is in this basis of eigenvectors that we will solve the differential equation. 
We will then perform inverse transformations (bijective) to obtain our physical quantities $ (h, v) $. 
The choice to solve the transport equation in the eigenvectors space is justified 
by the fact that the control is defined on the basis of these functions.


%%%%%%%%%%%%%%%%%%%%%%%%%%%%%%%%%%%%%%%%%%%%%%%%%%%%%%%%%%%%%%%%%%%%%%%%%%%%%%%%%%%%%%%%%%%%%%%%%%%%%%%%%%%%%%%%%%%%%%%%%%%%%%%%%
\section{Discretization}
%%%%%%%%%%%%%%%%%%%%%%%%%%%%%%%%%%%%%%%%%%%%%%%%%%%%%%%%%%%%%%%%%%%%%%%%%%%%%%%%%%%%%%%%%%%%%%%%%%%%%%%%%%%%%%%%%%%%%%%%%%%%%%%%%



\subsection{State}

We consider a uniform space grid $0 = x_0 < \cdots < x_{N_x} = L$, 
with a constant space step $h$.
By evaluating a function $f$ of $L^2(0,L)$ on this grid, 
we obtain the vector
$\tilde{f} = (f(x_i))_{0<i<N_x} $.

Then, for the transformed state $\xi = (\xi_1,\xi_2)$,
in order to take into account the condition on the water tank boundaries,
namely $x_0=0$ and $x_{N_x}=L$,
we will not compute the last value of the vector $\tilde{\xi_1}$ 
and the first value of the vector $\tilde{\xi_2}$.

Therefore, our numerical computation considers the concatenation of two vectors 
 
\[
\begin{cases}
	X_1 = ( \xi_1(h), \xi_1(2h),..., \xi_1(L))\\
	X_2 = (\xi_2(0), \xi_2(h),..., \xi_2(L-h)).
\end{cases}
\] 

At the end, we will artificially add the value of $ \ xi_1 (0) $ and $ \ xi_2 (L) $, 
just before going back into physical space.

Moreover, this choice of discretization implies 
that the operators are going to be defined in a dimension, 
which is even. May be problematic ?


\subsection{Operators}

In this section we show how we discretize the operator

\begin{equation}
	\label{def:A}
	\mathcal{A} = \Lambda \partial_x + \delta(x)J
\end{equation}

with 

\[
\Lambda = 
\left(
\begin{matrix}
	1 & 0 \\
	0 & -1 
\end{matrix}
\right), 
\quad
J =
\left(
\begin{matrix}
	0 & \frac{1}{3} \\
	-\frac{1}{3} & 0
\end{matrix}
\right), 
\]

and

\[ \delta (x) = -\frac{3}{4} \gamma (1+ \frac{1}{2}(L+x)) \]

The first term will be called the transport operator and the rotation operator.

\vspace{0.5cm}
\underline{Transport operator}

In order to discretize the derivative,
we use an explicit finite difference scheme. 
Since the transport speed is constant equal to one,
we place ourselves at each time step under the CFL condition
(the constant time step will be than determined 
with the parameters of the physical system).

We choose than to discretize $+\partial_x$ 
on our grid with an upstream scheme
and $-\partial_x$ 
with a downstream scheme.
If we denote $F$ the final matrix we obtain

\[
\begin{cases}
	FX_1 [i-1] = \partial_x \xi_1 (x_i) = \frac{X_1[i]-X_1[i-1]}{h} & i \in [2:N_x] \\
	FX_1 [0] = \partial_x \xi_1 (h) = \frac{X_1[1]-\xi_1(0)}{h} = \frac{X_1[1]+ X_2[0]}{h}
\end{cases}
\]

and

\[
\begin{cases}
	FX_2 [i] = \partial_x \xi_2 (x_i) = \frac{X_1[i+1]-X_1[i]}{h} & i \in [0:N_x-2] \\
	FX_2 [N_x-1] = \partial_x \xi_2 (L-h) = \frac{\xi_2(L)-X_2[N_x-1]}{h} = \frac{-X_1[N_x]- X_2[N_x-1]}{h}
\end{cases}
\]

The corresponding matrix is displayed below in the basis $(X_1,X_2)$

\[
F = \frac{1}{h}
\left(
\begin{array}{cccc|cccc}
	1  & 0  & \cdots & 0   &   +1 & \cdots & 0      & 0 \\
	-1 & 1  & \cdots & 0   &    0 & \ddots & 0      & 0\\
	1  & 0  & \ddots & 0   &    0 &        &        & 0\\
	0  & 0  & -1     & 1   &    0 &        & \ddots & 0\\ 
	\hline
    0  & 0  &        & 0   &   -1 &  1     &  0     & 0\\
	0  & 0  &        & 0   &   0  &  -1    &  1     & 0\\
	0  & 0  &        & 0   &   0  &  0     & \ddots & 0\\
	0  & 0  &        & -1  &   0  &        &  0     & -1\\
\end{array}
\right)
\]


\vspace{0.5cm}
\underline{Rotation operator}

This operator has an immediate discrete match. 
The only difficulty lies in the fact that the discretization of the function $ \ delta (x) $, 
and more particularly it will be necessary to pay attention to the elimination of the component corresponding to $ 0 $ for the first vector, 
and that corresponding to $ N_x $ for the second.

Thus, the final matrix denoted $ G $ is written

\[
G = \frac{1}{3}
\left(
\begin{array}{cccc|cccc}
	0 & 0 & \cdots & 0   &    \delta(0) & \cdots & 0      & 0 \\
	0 &   &        &     &    0 &  & \delta(h)      & 0\\
	  &   & \ddots &     &    0 &        &   \ddots & 0\\
	  &   &  0     & 0   &    0 &        &         & \delta(L-h)\\ 
	\hline
  - \delta(h) & 0          &        & 0            &   0  &  0     &  0     & 0\\
	       0  & -\delta(h) &        & 0            &   0  &  0     &  0     & 0\\
	          & 0          & \ddots &              &   0  &        & \ddots & 0\\
	      0   &            &        & -\delta(L)   &   0  &        &  0     & 0\\
\end{array}
\right)
\]




%%%%%%%%%%%%%%%%%%%%%%%%%%%%%%%%%%%%%%%%%%%%%%%%%%%%%%%%%%%%%%%%%%%%%%%%%%%%%%%%%%%%%%%%%%%%%%%%%%%%%%%%%%%%%%%%%%%%%%%%%%%%%%%%%
%%%%%%%%%%%%%%%%%%%%%%%%%%%%%%%%%%%%%%%%%%%%%%%%%%%%%%%%%%%%%%%%%%%%%%%%%%%%%%%%%%%%%%%%%%%%%%%%%%%%%%%%%%%%%%%%%%%%%%%%%%%%%%%%%
%%%%%%%%%%%%%%%%%%%%%%%%%%%%%%%%%%%%%%%%%%%%%%%%%%%%%%%%%%%%%%%%%%%%%%%%%%%%%%%%%%%%%%%%%%%%%%%%%%%%%%%%%%%%%%%%%%%%%%%%%%%%%%%%%
%%%%%%%%%%%%%%%%%%%%%%%%%%%%%%%%%%%%%%%%%%%%%%%%%%%%%%%%%%%%%%%%%%%%%%%%%%%%%%%%%%%%%%%%%%%%%%%%%%%%%%%%%%%%%%%%%%%%%%%%%%%%%%%%%
%%%%%%%%%%%%%%%%%%%%%%%%%%%%%%%%%%%%%%%%%%%%%%%%%%%%%%%%%%%%%%%%%%%%%%%%%%%%%%%%%%%%%%%%%%%%%%%%%%%%%%%%%%%%%%%%%%%%%%%%%%%%%%%%%
\end{document}